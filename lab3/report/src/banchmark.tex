\section{Тесты}

Чтобы убедиться, что AVL реализовано корректно, я 5000 раз добавлял элемент и потом убирал их в случайном порядке, и сравнивал количество элементов дерева с ожидаемым количеством.
А также, чтобы проверить, что дерево сбалансированное, я вывел высоту дерева из $10^6$ элементов.

Производительность своей реализации AVL, буду сравнивать с std::map

\begin{lstlisting}[language=C++]
#include <iostream>

#include <gtest/gtest.h>


#include <algorithm>

#include <chrono>
#include <random>


#include "avl.hpp"


std::mt19937 randomKey(std::chrono::steady_clock::now().time_since_epoch().count());


TEST(AvlTest, MyTest) {
    TAVlTree avl;
    avl.Insert(1);
    avl.Insert(2);
    EXPECT_EQ(2, avl.Size());
    avl.Clear();
    EXPECT_EQ(0, avl.Size());
}

TEST(AvlTest, CommonTest) {
    size_t const countActions = 5000;
    TAVlTree avl;
    std::vector <int> keys;
    for (size_t i = 0; i < countActions; ++i) {
        int key = static_cast<int>(randomKey());
        avl.Insert(key);
        keys.push_back(key);
        EXPECT_EQ(i + 1, avl.Size());
    }
    std::random_device randomDevice;
    std::mt19937 perm(randomDevice());
    std::shuffle(keys.begin(), keys.end(), perm);
    for (size_t i = 0; i < countActions; ++i) {
        int key = keys[i];
        EXPECT_EQ(avl.Exist(key), true);
        avl.Erase(key);
        EXPECT_EQ(countActions - i - 1, avl.Size());
    }
    EXPECT_EQ(avl.Empty(), true);
}

TEST(AvlTest, HeightTest) {
    size_t const countNodes = 1000000;
    TAVlTree avl;
    for (size_t i = 0; i < countNodes; ++i) {
        int key = static_cast<int>(randomKey());
        avl.Insert(key);
    }
    std::cout << "Height: " << avl.HeightTree() << std::endl;
}
\end{lstlisting}


\begin{lstlisting}[language=C++]
    #include <cstddef>
    #include <iostream>
    
    #include <gtest/gtest.h>
    
    
    #include <algorithm>
    
    #include <chrono>
    #include <random>
    #include <string>
    
    
    #include "avl.hpp"
    
    
    std::mt19937 randomKey(std::chrono::steady_clock::now().time_since_epoch().count());
    
    
    TEST(AvlTest, Benchmark) {
        const size_t countOfActions = 1000000;
        TAVlTree avl;
        std::map <std::string, unsigned long long> dict;
        std::vector <std::pair<std::string, unsigned long long>> dataInsert;
        std::vector <std::pair<std::string, unsigned long long>> dataErase;
        for (size_t i = 0; i < countOfActions; ++i) {
            std::string key = std::to_string(randomKey());
            unsigned long long value = randomKey();
            dataInsert.emplace_back(key, value);
            dataErase.emplace_back(key, value);
        }
        std::random_device randomDevice;
        std::mt19937 perm(randomDevice());
        std::shuffle(dataErase.begin(), dataErase.end(), perm);
    
    
        auto begin = std::chrono::high_resolution_clock::now();
        for (size_t i = 0; i < countOfActions; ++i) {
            auto key = dataInsert[i].first;
            auto value = dataInsert[i].second;
            avl.Insert(key, value);
        }
        for (size_t i = 0; i < countOfActions; ++i) {
            auto key = dataInsert[i].first;
            avl.Erase(key);
        }
        auto end = std::chrono::high_resolution_clock::now();
        auto timeAVL = std::chrono::duration_cast<std::chrono::milliseconds>(end - begin).count();
    
    
        begin = std::chrono::high_resolution_clock::now();
        for (size_t i = 0; i < countOfActions; ++i) {
            auto key = dataInsert[i].first;
            auto value = dataInsert[i].second;
            dict[key] = value;
        }
        for (size_t i = 0; i < countOfActions; ++i) {
            auto key = dataInsert[i].first;
            dict.erase(key);
        }
        end = std::chrono::high_resolution_clock::now();
        auto timeMAP = std::chrono::duration_cast<std::chrono::milliseconds>(end - begin).count();
    
    
        std::cout << "AVL time: " << timeAVL << " ms\n";
        std::cout << "std::map time: " << timeMAP << " ms\n";
    
    }
    
    

\end{lstlisting}


\begin{alltt}


4: Test command: /home/alex/mai-da-labs/build/tests/avl_test
4: Working Directory: /home/alex/mai-da-labs/build/tests
4: Test timeout computed to be: 10000000
4: Running main() from /home/alex/mai-da-labs/build/_deps/googletest-src/googletest/src/gtest_main.cc
4: [==========] Running 3 tests from 1 test suite.
4: [----------] Global test environment set-up.
4: [----------] 3 tests from AvlTest
4: [ RUN      ] AvlTest.MyTest
4: [       OK ] AvlTest.MyTest (0 ms)
4: [ RUN      ] AvlTest.CommonTest
4: [       OK ] AvlTest.CommonTest (149 ms)
4: [ RUN      ] AvlTest.HeightTest
4: Height: 37
4: [       OK ] AvlTest.HeightTest (1727 ms)
4: [----------] 3 tests from AvlTest (1877 ms total)
4: 
4: [----------] Global test environment tear-down
4: [==========] 3 tests from 1 test suite ran. (1877 ms total)
4: [  PASSED  ] 3 tests.
1/1 Test #4: avl_test .........................   Passed    1.88 sec

The following tests passed:
    avl_test

100% tests passed, 0 tests failed out of 1

Total Test time (real) =   1.89 sec
[alex@fedora build]$ ctest -V -R lab2_test
UpdateCTestConfiguration  from :/home/alex/mai-da-labs/build/DartConfiguration.tcl
UpdateCTestConfiguration  from :/home/alex/mai-da-labs/build/DartConfiguration.tcl
Test project /home/alex/mai-da-labs/build
Constructing a list of tests
Done constructing a list of tests
Updating test list for fixtures
Added 0 tests to meet fixture requirements
Checking test dependency graph...
Checking test dependency graph end
test 5
    Start 5: lab2_test

5: Test command: /home/alex/mai-da-labs/build/tests/lab2_test
5: Working Directory: /home/alex/mai-da-labs/build/tests
5: Test timeout computed to be: 10000000
5: Running main() from /home/alex/mai-da-labs/build/_deps/googletest-src/googletest/src/gtest_main.cc
5: [==========] Running 1 test from 1 test suite.
5: [----------] Global test environment set-up.
5: [----------] 1 test from AvlTest
5: [ RUN      ] AvlTest.Benchmark
5: AVL time: 4025 ms
5: std::map time: 2683 ms
5: [       OK ] AvlTest.Benchmark (7091 ms)
5: [----------] 1 test from AvlTest (7091 ms total)
5: 
5: [----------] Global test environment tear-down
5: [==========] 1 test from 1 test suite ran. (7091 ms total)
5: [  PASSED  ] 1 test.
1/1 Test #5: lab2_test ........................   Passed    7.10 sec

The following tests passed:
    lab2_test

100% tests passed, 0 tests failed out of 1

Total Test time (real) =   7.10 sec
    
\end{alltt}

Как видно, $std::map$ выиграл у $AVL$, так как $std::map$ реализовано с помощью КЧД, и, возможно, операция балансировки вызывается реже.


\pagebreak

