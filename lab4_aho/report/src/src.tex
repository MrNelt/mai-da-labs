\section{Описание}

Алгоритм Ахо-Корасик

Пусть дан набор строк и текст. Нужно найти все вхождения строк в текст.

Решать эту задачу будем следующим образом. Будем обрабатывать символы текста по одному и поддерживать наибольшую строку,
являющуюся наибольшим префиксом паттерна, и при этом также суффиксом считанного на данный момент текста. Если эта строка совпадает с каким-то
паттерном, то отметим текущий символ — в нём заканчивается какая-то строка.

Для этой задачи нам нужно как-то эффективно хранить и работать со всеми префиксами паттернами — для этого нам и понадобится префиксное дерево.

Добавим все слова в префиксное дерево и пометим соответствующие им вершины как терминальные.
Теперь наша задача состоит в том, чтобы при добавлении очередного символа быстро находить вершину
в префиксном дереве,
которая соответcтвует
max
входящему в бор суффиксу нового выписанного префикса. Для этого нам понадобятся несколько вспомогательных понятий.

1) связи неудач (они ведут ровно в те вершины, которые соответствуют самому длинному «сматченному» суффиксу)

2) связи выхода (они были в моей первой реализации алгоритма). Связи выхода необходимы, так как некоторые паттерны могут быть подстроками других паттернов.
Поэтому они просто ведут в терминальную вершину.

Алгоритм поиска:

Обрабатываем текст посимвольно. Пытаемся проходить по ветке бора, если вершина терминальная, то выписываем ответ,
а также обрабатываем рекурсивно связи выхода. Если пройти по вершине не получается, то проходим по связи неудач.

Алгоритм построения связей неудач:

для всех паттернов длины 1 - ведет в корень.
Для всех паттернов длины $n$ - идем по связи неудач длины $n - 1$(она точно есть), пытаемся пройти из этой вершины дальше,
если не получается, то идем дальше по связи неудач. Повторяем, пока не получится пройти дальше или не дойдем до корня.

Реализация с помощью BFS

Алгоритм построения связи выхода:

во время построения связей неудач проверяем:

1) если для текущей вершины связь неудачи ведет в терминальную вершину, то она является связью выхода.

2) если для текущей вершины связь неудачи ведет в нетерминальную вершину со связью выхода. То она так же является связью выхода
для текущей вершины.


\pagebreak

\section{Исходный код}

Cтруктура программы состоит из 2 файлов:

1. $aho.hpp$ Реализация алгоритма Ахо-Корасик.

2. $main\_cpp$ Обработка паттернов и текста, вывод ответа.

3. $CmakeLists.txt$ Описание сборки программы
\begin{lstlisting}[language=C++]
    #ifndef AHO_HPP
    #define AHO_HPP

    #include <queue>
    #include <unordered_map>
    #include <vector>
    #include <iostream>
    #include <sstream>

    class TAho {

    private:
        struct TNode {
            unsigned long long value;
            std::vector<unsigned long long> id;
            TNode* parent;
            TNode* fail;
            std::vector<size_t> size;
            std::unordered_map<unsigned long long, TNode*> go;
            bool isLeaf;
        };
        TNode* root;

        static void Destroy(TNode* node) {
            if (node != nullptr) {
                for (auto sons : node->go) {
                    Destroy(sons.second);
                }
                delete node;
            }
        }

        TNode* GetFailPointerFromLastFail(TNode* node, unsigned long long num) {
            if (node->go[num] == nullptr) {
                if (node != root) {
                    return GetFailPointerFromLastFail(node->fail, num);
                }
                return root;
            }
            return node->go[num];
        }

        TNode* GetFail(TNode* node, unsigned long long num) {
            if (node == root) {
                return root;
            }
            if (node->parent == root) {
                return root;
            }
            return GetFailPointerFromLastFail(node->parent->fail, num);
        }

        TNode* Next(TNode* node, unsigned long long num) {
            if (node == root) {
                if (node->go[num] != nullptr) {
                    return node->go[num];
                }
                return node;
            }
            if (node->go[num] != nullptr) {
                return node->go[num];
            }
            return Next(node->fail, num);
        }

    public:

        TAho() {
            root = new TNode;
        }

        ~TAho() {
            Destroy(root);
        }

        void AddVectorOfNums(std::vector<unsigned long long>& nums, unsigned long long pid) {
            auto *temp = root;
            for (auto num : nums) {
                if (temp->go[num] == nullptr) {
                    temp->go[num] = new TNode;
                    temp->go[num]->parent = temp;
                    temp = temp->go[num];
                    temp->value = num;
                } else {
                    temp = temp->go[num];
                }
            }
            temp->isLeaf = true;
            temp->id.push_back(pid);
            temp->size.push_back(nums.size());

        }


        void Init() {
            std::queue <TNode*> queue;
            queue.push(root);
            while (!queue.empty()) {
                auto* node = queue.front();
                queue.pop();
                node->fail = GetFail(node, node->value);
                if (node != root && node->fail->isLeaf) {
                    node->isLeaf = true;
                    for (size_t i = 0; i < node->fail->size.size(); ++i) {
                        node->size.push_back(node->fail->size[i]);
                        node->id.push_back(node->fail->id[i]);
                    }
                } else if (node != root && node->fail->fail != nullptr && node->fail->fail->isLeaf) {
                    node->isLeaf = true;
                    for (size_t i = 0; i < node->fail->fail->size.size(); ++i) {
                        node->size.push_back(node->fail->fail->size[i]);
                        node->id.push_back(node->fail->fail->id[i]);
                    }
                }
                for (auto elem : node->go) {
                    if (elem.second != nullptr) {
                        queue.push(elem.second);
                    }
                }
            }
        }

        void AddToAnswer(std::vector <std::pair<unsigned long long, unsigned long long>>& answer, TNode* temp, unsigned long long pos) {
            if (temp->isLeaf && temp != root) {
                for (size_t i = 0; i < temp->size.size(); ++i) {
                    answer.emplace_back(pos - temp->size[i], temp->id[i]);
                }
            }

        }

        std::vector <std::pair<unsigned long long, unsigned long long>> FindPosPatterns(const std::vector<unsigned long long >& nums) {
            std::vector <std::pair<unsigned long long, unsigned long long>> answer;
            auto* temp = root;
            unsigned long long pos = 0;
            for (const auto& num : nums) {
                AddToAnswer(answer, temp, pos);
                temp = Next(temp, num);
                pos++;
            }
            AddToAnswer(answer, temp, pos);
            return answer;
        }
    };

    #endif
\end{lstlisting}

\begin{lstlisting}[language=C++]
    #include <iostream>
    #include <string>
    #include <sstream>
    #include "aho.hpp"


    int main() {
        TAho aho;
        std::string line;
        std::getline(std::cin, line);
        int pid = 0;
        while (!line.empty()) {
            pid++;
            int num;
            auto stream = std::istringstream(line);
            std::vector<unsigned long long int> nums;
            while (stream >> num) {
                nums.push_back(num);
            }
            aho.AddVectorOfNums(nums, pid);
            std::getline(std::cin, line);
        }
        aho.Init();
        std::vector<std::pair<int, int>> info;
        std::vector<unsigned long long int> text;
        int page = 0;
        int num = 0;
        line = "";
        std::getline(std::cin, line);
        while (!line.empty()) {
            page++;
            auto stream = std::istringstream(line);
            num = 0;
            int elem;
            while (stream >> elem) {
                num++;
                text.push_back(elem);
                info.emplace_back(page, num);
            }
            std::getline(std::cin, line);
        }
        auto answers = aho.FindPosPatterns(text);;
        for (auto answer : answers) {
            std::cout << info[answer.first].first << ", " << info[answer.first].second << ", " << answer.second << "\n";
        }
    }

\end{lstlisting}
\begin{lstlisting}
    add_executable(lab4_aho main.cpp include/aho.hpp)
    target_include_directories(lab4_aho PRIVATE include)
\end{lstlisting}
\pagebreak

\section{Консоль}
\begin{alltt}
[alex@fedora build]$ cmake ../
-- The C compiler identification is GNU 12.2.1
-- The CXX compiler identification is GNU 12.2.1
-- Detecting C compiler ABI info
-- Detecting C compiler ABI info - done
-- Check for working C compiler: /usr/bin/cc - skipped
-- Detecting C compile features
-- Detecting C compile features - done
-- Detecting CXX compiler ABI info
-- Detecting CXX compiler ABI info - done
-- Check for working CXX compiler: /usr/bin/c++ - skipped
-- Detecting CXX compile features
-- Detecting CXX compile features - done
-- Configuring done (6.3s)
-- Generating done (0.0s)
-- Build files have been written to: /home/alex/mai-da-labs/build
[alex@fedora build]$ cmake --build .
[23/23] Linking CXX executable tests/avl_test
[alex@fedora build]$ cd lab4_aho/
[alex@fedora lab4_aho]$ cat test.txt
1 0002 001 2
1 2 1
2 02 2 2

01 2 1 2   1 2 1 3
2 2 2 2 02
[alex@fedora lab4_aho]$ ./lab4_aho <test.txt
1, 1, 2
1, 1, 1
1, 3, 2
1, 3, 1
1, 5, 2
2, 1, 3
2, 2, 3
\end{alltt}
\pagebreak

