\section{Описание}
Требуется написать реализацию алгоритма поразрядной сортировки.

Сам алгоритм состоит в последовательной сортировке объектов какой-либо устойчивой сортировкой по каждому разряду,
в порядке от младшего разряда к старшему, после чего последовательности будут расположены в требуемом порядке.
В качестве устойчивой сортировки буду использовать сортировку подсчётом.

\pagebreak

\section{Исходный код}
Формально структура программы состоит из 3 файлов:

1. $radix\_sort.h$ В нем содержится структура сортируемых объектов, сигнатура функции поразрядной сортировки

2. $radix\_sort.cpp$ В нем содержится реализация поразрядной сортировки

3. $main\_cpp$ Ввод объектов, сортировка объектов, вывод объектов
 
\begin{lstlisting}[language=C++]
#ifndef RADIX_SORT_H
#define RADIX_SORT_H

#include <iostream>
#include <string>
#include <vector>

struct TNode {
	static const size_t KEY_SIZE = 8;
	static const size_t DIGIT_SIZE = 128;
	std::string key;
	std::string *value;
	int GetValue(size_t n) {
		return static_cast<int>(key[n]);
	};
};


void RadixSort(std::vector <TNode>& data, size_t n);

#endif
	
\end{lstlisting}

\begin{lstlisting}[language=C++]
#include "radix_sort.h"


void RadixSort(std::vector<TNode> &data, size_t n) {
	for (size_t i = TNode::KEY_SIZE; i > 0; --i) {
		std::vector <TNode> result(n);
		std::vector <int> count(TNode::DIGIT_SIZE);
		for (size_t j = 0; j < n; ++j) {
			count[data[j].GetValue(i - 1)]++;
		}
		for (size_t j = 1; j < TNode::DIGIT_SIZE; ++j) {
			count[j] += count[j - 1];
		}
		for (size_t j = n; j > 0; j--) {
			int index = count[data[j - 1].GetValue(i - 1)];
			result[index - 1] = data[j - 1];
			count[data[j - 1].GetValue(i - 1)]--;
		}
		for (size_t j = 0; j < n; ++j) {
			data[j] = result[j];
		}
	}
}
\end{lstlisting}

\begin{lstlisting}[language=C++]
#include <iostream>

#include "radix_sort.h"


int main() {
	std::ios::sync_with_stdio(false);
	std::cout.tie(nullptr);
	std::cin.tie(nullptr);

	std::vector <TNode> data;
	std::string first;
	std::string number;
	std::string second;
	std::string value;
	while (std::cin >> first >> number >> second >> value) {
		TNode elem;
		elem.value = new std::string;
		std::string key;
		key += first;
		key += " ";
		key += number;
		key += " ";
		key += second;
		elem.key = key;
		*elem.value = value;
		data.push_back(elem);
	}
	if (!data.empty()) {
		RadixSort(data, data.size());
	}
	for (const auto & elem : data) {
		std::cout << elem.key << "\t" << *elem.value << "\n";
		delete elem.value;
	}
}
\end{lstlisting}
\pagebreak

\section{Консоль}
\begin{alltt}
[alex@fedora mai-da-labs]$ cd build
[alex@fedora build]$ cmake ../
-- Configuring done
-- Generating done
-- Build files have been written to: /home/alex/mai-da-labs/build
[alex@fedora build]$ cmake --build .
[ 21%] Built target lab1
[ 35%] Built target gtest
[ 50%] Built target gtest_main
[ 57%] Building CXX object tests/CMakeFiles/lab1_test.dir/lab1_test.cpp.o
[ 64%] Linking CXX executable lab1_test
[ 71%] Built target lab1_test
[ 85%] Built target gmock
[100%] Built target gmock_main
[alex@fedora build]$ cd lab1/
[alex@fedora lab1]$ ./lab1 
A 999 ZZ ZZZZZZZZZZZ
A 888 ZZ F
B 999 EE DDDDDDDD
H 991 FF SSS
A 888 ZZ	F
A 999 ZZ	ZZZZZZZZZZZ
B 999 EE	DDDDDDDD
H 991 FF	SSS

\end{alltt}
\pagebreak

